\documentclass[11pt]{article}
\usepackage[utf8]{inputenc}
\usepackage[slovene]{babel}
\usepackage{amsthm,amsfonts,amsmath,amssymb,url}
\usepackage{mathtools}
\usepackage{bm}
\usepackage{esvect}
\usepackage{bbm}


\textheight 210 true mm
\textwidth 146 true mm
\voffset=-17mm
\hoffset=-13mm

\newtheorem{Izrek}{{\sc Izrek}}[section]
\newtheorem{Trditev}[Izrek]{{\sc Trditev}}
\newtheorem{Posledica}[Izrek]{{\sc Posledica}}
\newtheorem{Definicija}[Izrek]{{\sc Definicija}}
\newtheorem{Zgled}[Izrek]{{\sc Zgled}}
\newtheorem{Opomba}[Izrek]{{\sc Opomba}}
\def\theIzrek{{\rm \arabic{section}.\arabic{Izrek}}}

\newenvironment{izrek}{\begin{Izrek}\sl}{\end{Izrek}}
\newenvironment{trditev}{\begin{Trditev}\sl}{\end{Trditev}}
\newenvironment{posledica}{\begin{Posledica}\sl}{\end{Posledica}}
\newenvironment{definicija}{\begin{Definicija}\rm }{\end{Definicija}}
\newenvironment{zgled}{\begin{Zgled}\rm }{\end{Zgled}}
\newenvironment{opomba}{\begin{Opomba}\rm }{\end{Opomba}}

\newenvironment{dokaz}[1][{\sc Dokaz}]{\begin{proof}[#1]\renewcommand*{\qedsymbol}{\(\blacksquare\)}}{\end{proof}}

\newcommand{\Mod}[1]{\hbox{ (mod } #1)}

\begin{document}
	
	\thispagestyle{empty}
	\begin{center}
		\begin{Large}
			{\bf Zapiski pri predmetu Verjetnost}
		\end{Large}
		
	\end{center}
	Hvala ker bereš.
	\vfill
	\begin{center}
		Ljubljana, 2017 $\quad \quad $ Gregor Vavdi
	\end{center}
	\newpage
	\setcounter{page}{1}
	
	%%%%%%%%%%%%%%%%%%%%%%%%%%%%%%%%%%%%%%%%%%%Verjetnost%%%%%%%%%%%%%%%%%%%%%%%%%%%%%%%%%%5
	\section{Osnovna verjetnost}
	\begin{Definicija}
		Klasična definicija verjetnosti:
		\\
		Naj bo  $A\subseteq \Omega $ dogodek, n število ugodnih izidov in N vsi možni izidi. Potem je verjetnost definirana:
		$$P(A) = \frac{n}{N}$$
	\end{Definicija}

	\begin{Izrek}
		Naj bo $\Omega$ prostor izidov verjetnostnega eksperimenta. Dogodki tvorijo družino $\mathcal{F} \subseteq 2^\Omega = P(\Omega)$ z lastnostmi:
		\begin{enumerate}
			\item
			$\Omega \in \mathcal{F}$
			\item
			$ A \in \mathcal{F} \Rightarrow A^C \in \mathcal{F} \quad (\Rightarrow \emptyset \in \mathcal{F})$
			\item
			$A_1, A_2,\dots \text{števen nabor dogodkov iz } \mathcal{F}$. Potem je $\bigcup A \in \mathcal{F}$
		\end{enumerate}
	\end{Izrek}
	
	\begin{Trditev}
		Računanje z dogodki:
		\begin{itemize}
			\item
			$A\cap A = A \quad A\cup A = A$
			\item
			$(A\cup B)\cup C = A\cup (B \cup C) \quad (A\cap B) \cap C = A \cap (B\cap C)$
			\item
			$(A\cap B)\cup C = (A \cup C ) \cap (B\cup C)$ 
			\item
			$(A\cup B)\cap C = (A \cap C ) \cup (B\cap C)$ 
			\item
			$(A \cup B)^C = A^C \cap B^C$
			\item
			$(A\cap B) ^C = A^C \cup B^C$
		\end{itemize}
		
		Dogodka A in B sta \textbf{nezdružljiva}, če je $A \cap B = \emptyset$.
	\end{Trditev}
		
	\begin{Definicija}
		Verjetnost je preslikava $\mathcal{F}\to [0,1]$, ki zadošča lastnosti:
		\begin{itemize}
			\item
			$P(\Omega) = 1$
			\item
			$P(A^C) = 1 - P(A) \quad (\Rightarrow P(\emptyset) = 0)$
			\item
			Naj bodo $(A_i)_{i \ge 1} $ paroma nezdružljivi dogodki $(A_i \cap A_j = 0)_{i\neq j}$.
			\item
			$P(\cup_{i =1}^{\infty}A_i) = \sum_{i = 1}^{\infty}P(A_i)$
		\end{itemize}
	\end{Definicija}
	
	\begin{Trditev}
		Princip vključevanja in izključevanja:
		$$P(A\cup B  ) = P(A) + P(B) - P(A \cap B)$$
		$$P(A \cup B \cup C) = P(A) + P(B) + P(C) - P(A \cap B) - P(A \cap C ) - P(B \cap C) + P(A\cap B \cap C)$$
	\end{Trditev}
	
	\begin{Trditev}
		Zveznost verjetnosti
		\\
		Če je $A_1 \supseteq A_2 \supseteq \dots \supseteq A_k \supseteq \dots $ padajoče zaporedje dogodkov je 
		\[
		P(\bigcap_{k=1}^{\infty} A_k) = \lim\limits_{k \to \infty}{P(A_k)}
		\]
	\end{Trditev}
	\begin{Trditev}
		Če je $ B_1 \subseteq B2 \subseteq ... \subseteq B_k \subseteq \dots$ naraščajoče zaporedje dogodkov, velja:
		\[
		P(\bigcup_{k=1}^{\infty} B_k) = \lim\limits_{k \to \infty}{P(B_k)}
		\]
	\end{Trditev}
	%--------------------------------------------------------------------------------------------------------------------
	\section{Pogojna verjetnost}
	\begin{Definicija}
		Naj bo dogodek $A \subseteq \omega \quad P(A) > 0$.
		Potem je $B\in \mathcal{F}$ definiramo $$P(A|B) = \frac{P(A \cap B)}{P(A)} \Rightarrow P(A\cap B) = P(A) \cdot P(A|B)$$
		$P(A|B)$ pravimo pogojna verjetnost in jo beremo kot: verjetnost dogodka A pri pogoju B.
	\end{Definicija}
	
	\subsection{Večfazni ali relejski poskusi}
	\subsection{Formula za popolno verjetnost}
	\begin{Definicija}
		$H_1,H_2,H_3,...$ (končno mnogo) naborov dogodkov. $\mathcal{F}$ \textbf{je popoln sistem dogodkov}, če velja:
		\begin{itemize}
			\item
			$H_i \cap H_j = \emptyset \quad i \ne j$
			\item
			$ P(\bigcup_{i=1}H_i) = \Omega$ (Unija vseh dogodkov je cel prostor)
			\item
			$P(H_i) > 0 \quad \forall i$	
		\end{itemize}
	\end{Definicija}
	\begin{Izrek}
		Formula za popolno verjetnost je 
		$$P(A) = \sum_{i = 1}{P(A|H_i) \cdot P(H_i)}$$
	\end{Izrek}
	\begin{Izrek}(Baynsova formula)
		$$ P(H_j|A) = \frac{P(H_j\cap A)}{P(A)}$$
	\end{Izrek}
	\begin{Izrek}
		Naj bosta dogodka $A$ in $B \in \mathcal{F}$. Dogodka sta \textbf{neodvisna}, če velja $$P(A\cap B) = P(A) \cdot P(B)$$
	\end{Izrek}
	\begin{Posledica}
		Naj bosta A in B neodvisna dogodka. Potem so neodvisni tudi naslednji dogodki:
		\begin{itemize}
			\item
			$\{A^c, B\}$
			\item
			$\{A, B^c\}$
			\item
			$\{A^c, B^c\}$
		\end{itemize}
	\end{Posledica}
	\begin{Posledica}
		$A,B,C \in \mathcal{F}$ neodvisne $ \Rightarrow P(A\cap B \cap C) = P(A) \cdot P(B) \cdot P(C)$
		\\
		V splošnem $P(A_{i_1} \cap A_{i_2} \cap ...\cap A_{i_k}) = P(A_{i_1})\cdot P(A_{i_2}) \cdot ... \cdot P(A_{i_k})$
	\end{Posledica}
	
	\section{Diskretne slučajne spremenljivke}
	\begin{Definicija}
		Naj bo $(\Omega, \mathcal{F}, P) $ verjetnostni prostor in funkcija $X: \Omega \to \mathbb{R}$, ki lahko zavzema kvečejmu števno mnogo vrednosti $w_1, w_2,\dots $. Naj bo $H_i$ množica vseh izidov za katere je vrednost funkcije enaka $x_i$. Povedano drugače:
		$H_i = \{ \omega_i \quad;\quad X(\omega) = x_i\}$. \textbf{$X$ je slučajna spremenljivka}, če je $H_i \in F \quad \forall i$.
		\\
		\[
		\frac{X}{p} = \begin{pmatrix}
		x_1 & x_2 & x_3 & \cdots & \\       p_1 & p_2 & p_3 & \cdots & 
		\end{pmatrix}
		\]
		$$p_i \ge 0 \quad \sum_{i=1}^{n}{p_i} = 1$$
	\end{Definicija}
	
	\subsection{Binomska porazdelitev}
	\begin{Definicija}
		Verjetnost, da bomo v n poskusih k-krat videli izzid p.
		\\
		Oznaka: \textbf{Bin(n,p)}.
	\end{Definicija}
	\begin{Definicija}
		$$P(X=k) = \binom{n}{k} p^k (1-p)^{n-k}$$
	\end{Definicija}
	\begin{Posledica}
		$$E(X) = n\cdot p$$
		$$Var(X) = n\cdot p\cdot q$$
	\end{Posledica}
	\subsection{Bernulijeva porazdelitev}
	\begin{Definicija}
		Indikator dogodka. \textbf{Be(p)}
		\[
		\frac{X}{p} = \begin{pmatrix}
		0 & 1 \\       1-p & p  
		\end{pmatrix}
		\]
	\end{Definicija}
	\begin{Posledica}
		$$E(X) = p$$
		$$Var(X) = p\cdot q$$
	\end{Posledica}
	\subsection{Geometrijska porazdelitev}
	\begin{Definicija}
		Število poskusov do prvega uspešnega izzida. \textbf{Geom(p)}.
	\end{Definicija}
	\begin{Definicija}
		$$P(X=k) = p \cdot q^{k-1}$$
	\end{Definicija}
	\begin{Posledica}
		$$E(X) = \frac{1}{p}$$
		$$Var(X) = \frac{q}{p^2}$$
	\end{Posledica}
	\begin{Opomba}
		Geometrijska porazdelitev nima spomina!
	\end{Opomba}
	\subsection{Poissonova porazdelitev}
	\begin{Definicija}
		Število telefonskih klicev, nesreč v določenem času. \textbf{Po($\lambda$)}.
	\end{Definicija}
	\begin{Definicija}
		$$P(X=k) = \frac{\lambda^k \cdot e^{-\lambda}}{k!}$$
	\end{Definicija}
	\begin{Posledica}
		$$E(X) = \lambda$$
		$$Var(X) = \lambda$$
	\end{Posledica}
	\subsection{Matematično upanje}
	\begin{Definicija}
		Naj bo $X$ slučajna spremneljivka s porazdelitveno shemo:
		\[
		\frac{X}{p} = \begin{pmatrix}
		x_1 & x_2 & x_3 & \cdots & x_k \\       p_1 & p_2 & p_3 & \cdots & p_k
		\end{pmatrix}
		\]
		\textbf{Matematično upanje} je:
		$$ E[X] = \sum_{k}{x_k \cdot p_k}$$
		pod pogojem, da je $\sum_{k}{|x_k|\cdot p_k} < \infty$
	\end{Definicija}
	\begin{Izrek}
		Naj bo X,Y slučajni spremenljivki. Tedaj velja:
		\begin{itemize}
			\item
			$ E[X]$ obstaja natanko tedaj, ko obstaja $ E[|X|]$. Pri tem velja $|E[X]| \le E[|X|]$.
			\item
			$ E[aX+b] = a \cdot  E[X] + b$
			\item
			$0 \le X \le Y \Rightarrow  E[X] \le  E[Y]$
			\item
			Če obstaja $ E[X]$ in $ E[Y]$, takrat velja: $ E[X+Y] =  E[X] +  E[Y]$ 
		\end{itemize}
	\end{Izrek}
	\subsection{Pogojno matematično upanje in pogojna porazdelitev}
	\begin{Definicija}
		Naj bo X diskretna slučajna spremenljivka in $A\in \mathcal{F}$, za katero velja $P(A) >0$.\textbf{ Pogojna porazdelitev $X$ pri pogoju A}:
		\[
		\frac{X}{p} = \begin{pmatrix}
		x_1 & \cdots & x_k \\       p_1 & \cdots & p_k
		\end{pmatrix}
		\]
		$$\textnormal{Pri čemer je}\quad p_k= \frac{P(X_k =x_k \cap A)}{P(A)}$$
	\end{Definicija}
	
	\begin{Definicija}
		Pogojno matematično upanje:
		$$E[X|A] = \sum_{i}{x_i \cdot P(X=x_i|A)} = \sum_{i}{x_i \cdot \frac{P(\{X_k =x_k\} \cap A)}{P(A)}} = \frac{E[X\cdot \mathcal{U}_A]}{P(A)}$$
	\end{Definicija}
	\begin{Definicija}
		\textbf{(Formula za popolno matematično upanje)} \\
		Naj bodo $A_1,A_2, \dotsc$ paroma tuji dogodki in $P(\bigcup_{i=1}A_i) = 1$. Potem za vsako slučajno spremenljivko X velja $$E[X] = \sum_{i}E[X|A_i] \cdot P(A_i)$$ 
	\end{Definicija}
	\section{Slučajni vektorji}
	\begin{Definicija}
		slučajni vektor je n-terica slučajnih spremenljvik: $X = (x_1, x_2,\dots , x_n)$
		$$ \vv{\bm{x}}: \Omega \to \mathbb{R}^d$$
		Porazdelitev posamične koordinate rečemo \textbf{robna porazdelitev}.
	\end{Definicija}
	\begin{Definicija}
		Slučajne spremenljivke $X_1,X_2, \dots , X_n$ so neodvisne, če velja:
		$$P(X_1 = x_1, X_2 = x_2,\dotsc, X_n = x_n) = P(X_1 = x_1) \cdot P(X_2 = x_2) \cdot \dotsc \cdot P(X_n = x_n)$$
	\end{Definicija}
	\begin{Posledica}
		Naslednje trditve so posledica zgornje definicije:
		\begin{itemize}
			\item
			$\forall A_1,\dotsc, A_n \subseteq \mathbb{R} \Rightarrow P(X_1 \in A_1, \dotsc, X_n\in A_n) = P(X_1 \in A_1) \cdot \dotsc \cdot P(X_n \in A_n)$
			\begin{Opomba}
				$$P(X\in A ) = \sum_{x\in A}{P(X = x)}$$
			\end{Opomba}
			\item
			Če so $X_1, \dotsc ,X_n$ neodvisni, potem so za $\forall \quad 1\le i_1 \le i_2 \le \dotsc \le i_k\le n$, tudi $X_{i_1}, X_{i_2},\dotsc, X_{i_n}$ neodvisni.
			\item
			Če so $X_1, \dotsc ,X_n$ neodvisne, so za vsak $ A_1,\dotsc, A_n \in \mathbb{R}$ dogodki
			\\
			$\{X_1\in A_1\}\dotsc \{X_n\in A_n \}$ neodvisni.
			\item
			Dogodki $B_1, \dotsc, B_n \in \mathcal{F}$ so neodvisni natanko tedaj, ko so slučajne spremenljivke $ \mathcal{U}_{B_1},\dotsc, \mathcal{U}_{B_n}$ neodvisne.
			\item
			Če so $X_1, \dotsc, X_n$ neodvisne in $f_1,\dotsc,f_n : \mathbb{R} \to \mathbb{R}$ so tudi $f_1(X_1),\dotsc, f_n(X_n)$ neodvisne.
			\item
			Denimo, da sta X in Y neodvisna. Potem velja:
			$$E[XY] = E[X] \cdot E[Y]$$
		\end{itemize}
	\end{Posledica}
\section{Splošne slučajne spremenljivke}
\begin{Definicija}
	Naj bo verjetnostni prostor definiran kot: $(\Omega, \mathcal{F},P)$. Nadalje naj bo $X: \Omega \to \mathbb{R}$. Zahteva, da je X slučajna spremenljivka : $\{ X\le x\} = \{\omega ; X(\omega) \le x\} \in \mathcal{F}$. Potem je $$F_X(x)  = P(X \le x)$$ dobro definirana funkcija $\mathbb{R}\to \mathbb{R}$ in jo imenujemo \textbf{porazdelitvena funkcija}.
\end{Definicija}
\begin{Trditev}
	Lastnosti porazdelitvene funkcije:
	\begin{itemize}
		\item
		$F_X(x) \in [0,1] \quad \forall x\in \mathbb{R}$
		\item
		$\{X \le x\} \subseteq \{X \le y \} \Rightarrow P(X \le x) \le P(X \le y) $
		\item
		$\lim\limits_{t \downarrow x}{F_X(t)} = F_x(x) \quad$ (zvezna iz desne).
		\item
		$F_X(x-) = \lim\limits_{t\uparrow x}F_X(t) \le F_X(x) \quad$
		(ker je $F_X(x)$ nepadajoča ima X v vsaki točki x tudi levo limito).
		
		\item
		Če velja $F_X(x -) = F_X(x)$, potem je $F_X(x)$ zvezna v točki $x\in\mathbb{R}$, sicer pa ima skok v točki $x\in \mathbb{R}$.
		\item
		Točk, kjer je porazdelitvena funkcija nezvezna (ima skok) je kvečejmu števno mnogo.
		\item
		$\lim\limits_{x \to \infty}{F_X(x)} = 1$
	\end{itemize}
\end{Trditev}
\begin{Definicija}
	Slučajna spremneljivka X ima\textbf{ gostoto} $\mathbf{p_X(x)}$, če velja:
	$$P(X\le x) = F_X(x) = \int_{-\infty}^{x}{p_X(t)dt}$$ za vsak $x\in \mathbb{R}$.
	\\
	Takim spremenljivkam pravimo absolutno zvezne slučajne spremneljivke. 
\end{Definicija}
\begin{Trditev}
	Če obstaja gostota, potem velja:
	$$\left( F_X(x) \right)' = \left(\int_{-\infty}^{x}{p_X(t)dt}\right)'= p_X(x)$$
\end{Trditev}
\begin{Trditev}(Lastnosti gostote)
\begin{itemize}
	\item
	$p_X(x) \ge 0$
	\item	
	$\int_{-\infty}^{\infty}{p_X(t) dt} = 1 $
\end{itemize}
\end{Trditev}
\subsection{Enakomerno zvezna na [a,b] }
\begin{Definicija}
	Naj bo X enakomerno zvzena slučajna spremenljivka. $X \sim EZ[a,b]$. Potem je gostota enaka:
	$$ p_X(x) = \left \{\begin{array}{lr}
	\frac{1}{b-a} : a\le x\le b \\ 0 :\quad \textnormal{sicer}
	  \end{array}
	  \right.
	$$
	$$E[X] = \frac{a+b}{2}$$
	$$Var(X) = \frac{(b-a)^2}{12}$$
\end{Definicija}
\subsection{Eksponentna porazdelitev}
\begin{Definicija}
	Naj bo X eksponentna slučajna spremeneljivka ($X\sim Exp(\lambda) \quad \lambda >0$). Definirana je kot čas čakanja na dogodek.
	Njena porazdelitvena funkcija je:
		$$ F_X(x) = \left \{\begin{array}{lr}
		1 - e^{-\lambda x}:  x>0 
		\\
		0 :\quad \textnormal{sicer}
		\end{array}\right.$$
		
		$$ p_X(x) = \left \{\begin{array}{lr}
		\lambda\cdot e^{-\lambda x} : x>0
		\\
		0 :\quad \textnormal{sicer}
		\end{array}
		\right.
		$$
		
		$$E[X] = \frac{1}{\lambda}$$
		
		$$Var(X) = \frac{1}{\lambda^2}$$
\end{Definicija}
\begin{Opomba}
	Eksponentna porazdelitev je brez spomina!
\end{Opomba}
\subsection{Standardna normalna porazdelitev ali Gaussova porazdelitev}
\begin{Definicija}
	Naj bo X standardno normalna slučajna spremneljivka $X\sim \mathcal{N}(0,1)$.
	Njena gostota je enaka:
	$$p_X(x) = \frac{1}{\sqrt{2\pi}}\cdot \mathrm{e}^{-\frac{1}{2}}(\frac{x - \mu}{\sigma})^2$$
	
	$$E[X] = \mu$$
	
	$$Var(X) = \sigma$$
\end{Definicija}
\subsection{Porazdelitev slučajnih spremenljivk}
\begin{Definicija}
	Naj bo $(X,Y) : \Omega \to \mathbb{R}^2$ slučajni vektor.
	$$F_{X,Y}(x,y) = P(X\le x, Y\le y) = \int_{-\infty}^{x}\int_{-\infty}^{y}{p_{X,Y}(x,y) dy dx}$$
\end{Definicija}
\begin{Definicija}
	$$\int_{-\infty}^{\infty}\int_{-\infty}^{\infty}{p_{X,Y}(x,y) dy dx} = 1$$
\end{Definicija}
\begin{Trditev}
	Slučajni spremenljivki X in Y sta neodvisni, če velja:
	$$P(X\le x, Y\le y) =F_{X,Y}(x,y) =F_X(x) \cdot F_Y(y) \quad \forall x,y\in \mathbb{R}$$
	$$p_{X,Y} (x,y) =p_X(x) \cdot p_Y(y) \quad \forall x,y\in \mathbb{R}$$
\end{Trditev}
\begin{Trditev}
	Naj bosta X in Y neodvisni slučajni spremenljivki. Nadalje naj velja $Z = X + Y$. Gostota slučajne spremenljivke Z je:
	$$p_Z(z) = \int_{-\infty}^{\infty}{dx}\left(\int_{-\infty}^{z-x}{p_{X,Y}(x,y) dy}\right) = \int_{-\infty}^{\infty}{p_X(x)\cdot p_Y(z-x)dx}$$
\end{Trditev}
\subsection{Matematično upanje zveznih slučajnih spremenljivk}
\begin{Definicija}
Će ima slučajna spremenljivka X gostoto $p_X(x)$ je $$E[X] = \int_{-\infty}^{\infty}{x\cdot p_X(x) dx}$$  pod pogojem da $\int_{-\infty}^{\infty}{|x| \cdot p_X(x) dx} < \infty$.
\end{Definicija}
\begin{Izrek}
Naj bo X slučajna spremenljivka in $p_X(x)$ gostota. Nadalje naj bo $f: \mathbb{R} \to \mathbb{R}$ zvezna, razen morda v končno mnogo točkah. Iz tega sledi, da je $f(X)$ slučajna spremenljivka, njeno matematično upanje je:
$$E[f(X)] =  \int_{-\infty}^{\infty}{f(x)\cdot p_X(x) dx}$$
pod pogojem da integral absolutno konvergira.
\end{Izrek}
\subsection{Pogojna porazdelitvena funkcija}
\begin{Definicija}
Naj bo X zvezna slučajna spremenljivka in naj bo A dogodek iz $\mathcal{F}$. Pogojna porazdelitvena funckija je definirana kot:
$$F_{X|A}(x) = P(X\le x | A) = \frac{P(X\le x, A)}{P(A)}$$
\end{Definicija}
\begin{Definicija}
Naj bo X zvezna slučajna spremenljivka in naj bo A dogodek iz $\mathcal{F}$. Potem obstaja pogojna gostota definiran kot odvod pogojne porazdelitvene funkcije $F_{X|A}$
\end{Definicija}
\begin{Definicija}
Če sta X in Y slučajni spremenljivki na $(\Omega,\mathcal{F},P)$, potem obstaja pogojna porazdelitev X glede na Y ( $  P(X \le x | Y \le y) $ ) za skoraj vsak $y\in \mathbb{R}$. Pogojno gostota je tako:
$$p_{X|Y}(x,y) = \frac{p_{X,Y}(x,y)}{p_X(x)}$$
pri čemer je $$p_X(x) = \int_{-\infty}^{\infty} p_{X,Y}(x,y) dx$$ 
\end{Definicija}
\subsection{Mediana}
\begin{Definicija}
Naj bo X slučajna spremenljvka in $a\in \mathbb{R}$. Število $a$ je \textbf{mediana}, če velja: $$P(X\le a) = \frac{1}{2}$$
\end{Definicija}
\subsection{Varianca}
\begin{Definicija}
Varianca ali disprezija je mera, kako slučajna spremenljivka oscilira okrog svoje pričakovane vrednosti. $$Var(X) = E [( X - E[X])^2] = E[X^2] - E[X]^2$$
\end{Definicija}
\begin{Definicija}
(Standardni odklon):
$$\sigma = \sqrt{Var(X)}$$
\end{Definicija}
\begin{Trditev}
Naj bo X slučajna spremenljivka in naj $Var(X)$ obstaja. Potem sledi:
\begin{itemize}
\item
$Var(X) \ge 0$ in $Var(X) = 0 \iff x = konst.$
\item
$Var(X + b) = Var(X)$
\item
$Var(a \cdot X) = a^2 \cdot Var(X)$
\end{itemize}
\end{Trditev}
\subsection{Kovarianca}
\begin{Definicija}
Naj bosta X in Y slučajni spremenljivki na $(\Omega,\mathcal{F},P)$. Nadalje naj obstaja $E[X], E[Y]$, ter $E[XY]$. \textbf{Kovarianca} je:
$$Cov(X,Y) = E[XY] - E[X] \cdot E[Y]$$
\end{Definicija}
\begin{Trditev} 
Naj bojo X,Y,Z slučajne spremenljivke na  $(\Omega,\mathcal{F},P)$. Nadlaje naj velja, da sta števili $a$ in $b\in \mathbb{R}$. Potem velja:
\begin{itemize}
\item
$Cov(a\cdot X + b\cdot Y, Z) = a \cdot Cov(X,Z) + b\cdot Cov(Y,Z)$
\item
$|Cov(X,Y)| \le \sigma (X) + \sigma (Y)$
\item
$Cov(X,X) = Var(X)$
\end{itemize}
\end{Trditev}
\begin{Izrek}
Naj bosta X in Y neodvisni (z drugim momentom). Takrat velja:
$$Cov(X,Y) = E[XY] - E[X] \cdot E[Y] = 0$$
\end{Izrek}
\begin{Posledica}
Naj bosta X in Y neodvisni. Takrat velja:
$$Var(X + Y) = Var(X) + Var(Y)$$
\end{Posledica}
\begin{Definicija}
(Korelacijski koeficient)
$$\rho  = \frac{Cov(X,Y)}{\sigma (X) \cdot \sigma (Y)}$$
\end{Definicija}

\end{document}

